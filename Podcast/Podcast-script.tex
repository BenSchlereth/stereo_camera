\documentclass[10pt,a4paper]{article}
\usepackage[latin1]{inputenc}

\title{Finding the Convex Hull with Application to Distance Measurements for a Formula Student Car}

\begin{document}
	\maketitle
	
	\section{Introduction}
	Hi
	My name is Benedikt Schlereth and I'm studying Applied Math and Physics in the sixth Bachelorsemester. 
	Right now I'm working for the Fraunhofer-Institut IIS here in Nuremberg, where we want to find the best feature-extractions tools to detect certain behaviors.
	To learn the practical implementation of new technology I support the local Formula-Student-Team Strohm und Soehne by designing PCB-Boards for our Electric car.
	In the future I also want to help out in our Software department.
	
	
	\section{motivation}
	First let me explain what Formula Student is:
	Formula-Student is an international design and construction competition where teams from all around the world are constructing and then racing single seat formula race-cars.
	With the new event of Formula-Student-Driver-less Teams are encouraged to develop their own self-Driving cars.
	My motivation is to kickoff the development of our own Driver-less Vehicle by implementing the first step and providing a way to detect the boundaries of a race track. 
	
	The car will drive on a set course marked with blue and yellow cones. So the only objects that need to be detected are the said cones and no other cars or people. There a strict safety-Rules in place, to protect man and machine which aren't of any interest for this project. 
	
	\section{starting}
	A complete self-Driving car would be a project for a big group or a PHD-Student, so I will focus only on detecting the edge of the track. 
	To detect the boundaries a computer has to visualize the cones.
	My first step is to extract only the important colors from a given image to filter out the background.
	So with only the racecar on the track the only blue or yellow parts should be the pins.
	Now the image should only consist of the cones that due to Image noise have odd shapes and no clear edges. 
	
	\section{Algorithmus}
	To get a clearer Image with sharper edges to examine I have to calculate the Convex hull of the remaining objects inside the Image.
	The Convex Hull is the smallest region containing all points while having no Concave corners.
	It is an important step for a few image-processing Algorithms, so development for  Gift-Wrapping-Algorithms are dating back quit far in computer history.	
	To lose not to much time on calculation it is important to choose an fast algorithm.
	There are several available and one is already implemented in OpenCV, an Open Source Python-Library for computer-Vision.
	But there are several drawbacks with Gift-wrapping algorithms.
	Most of them are slow and the fast ones are proved to be wrong on some special occasions.
	
	\section{publication}
	And now lets have a look on one algorithm by Graham and Yao.
	
	\section{wrap-up}
	
\end{document}